% resume.tex
%
% (c) 2002 Matthew Boedicker <mboedick@mboedick.org> (original author) http://mboedick.org
% (c) 2003-2007 David J. Grant <davidgrant-at-gmail.com> http://www.davidgrant.ca
%
%This work is licensed under the Creative Commons Attribution-Noncommercial-Share Alike 2.5 License. To view a copy of this license, visit http://creativecommons.org/licenses/by-nc-sa/2.5/ or send a letter to Creative Commons, 543 Howard Street, 5th Floor, San Francisco, California, 94105, USA.

\documentclass[letterpaper,11pt]{article}

%-----------------------------------------------------------
%Margin setup

\setlength{\voffset}{0.1in}
\setlength{\paperwidth}{8.5in}
\setlength{\paperheight}{11in}
\setlength{\headheight}{0in}
\setlength{\headsep}{0in}
\setlength{\textheight}{11in}
\setlength{\textheight}{9.5in}
\setlength{\topmargin}{-0.25in}
\setlength{\textwidth}{7in}
\setlength{\topskip}{0in}
\setlength{\oddsidemargin}{-0.25in}
\setlength{\evensidemargin}{-0.25in}
%-----------------------------------------------------------
%\usepackage{fullpage}
\usepackage{shading}
%\textheight=9.0in
\pagestyle{empty}
\raggedbottom
\raggedright
\setlength{\tabcolsep}{0in}

\usepackage{hyperref}
\usepackage{fontspec}
\usepackage{xunicode}
\usepackage{xltxtra}
\XeTeXlinebreaklocale "zh"
\XeTeXlinebreakskip = 0pt plus 1pt
\setsansfont{Vera Sans YuanTi Mono}
\setromanfont{Vera Sans YuanTi Mono}

%-----------------------------------------------------------
%Custom commands
\newcommand{\resitem}[1]{\item #1 \vspace{-2pt}}
\newcommand{\resheading}[1]{{\large \parashade[.9]{sharpcorners}{\textbf{#1 \vphantom{p\^{E}}}}}}
\newcommand{\ressubheading}[4]{
\begin{tabular*}{6.5in}{l@{\extracolsep{\fill}}r}
		\textbf{#1} & #2 \\
		\textit{#3} & \textit{#4} \\
\end{tabular*}\vspace{-6pt}}
%-----------------------------------------------------------


\begin{document}

\begin{tabular*}{7in}{l@{\extracolsep{\fill}}r}
\textbf{\Large 盛 艳}  & 13813147410\\
\#江苏省扬州大学 &  shengyan1985-at-gmail.com \\
信息工程学院 & http://liz.appspot.com\\
\end{tabular*}
\\

\vspace{0.1in}

\resheading{基本资料}
\begin{itemize}
\item
\item {} 
性别: 女

\item {} 
出生年月: 1985年8月

\item {} 
籍贯: 苏州

\item {} 
健康状况: 良好

\item {} 
毕业院校: \href{http://www.yzu.edu.cn}{扬州大学} 信息工程学院

\item {} 
专业/方向: 计算机应用技术/数据挖掘

\item {} 
通信地址: 江苏省扬州大学信息工程学院400173\#
\end{itemize}

\resheading{英语水平}
\begin{itemize}
\item {} 
于2004年获得CET-4证书;

\item {} 
具有扎实的英语应用能力及较强的计算机专业英文文献阅读及翻译水平, 并已发表多篇英文学术论文.

\end{itemize}

\resheading{计算机能力/技能}
\begin{itemize}
\item {} 
于2004年获得CET-4证书;

\item {} 
具有扎实的英语应用能力及较强的计算机专业英文文献阅读及翻译水平, 并已发表多篇英文学术论文.

\end{itemize}

\resheading{School Projects}
\begin{itemize}
\item
	\ressubheading{SpectraVu Medical}{Vancouver, BC}{Engineering Physics Project Lab, APSC 479}{Sep. 2001 - Apr. 2002}
	\begin{itemize}
		\resitem{	Designed and implemented a digital video processing system for lung cancer imaging,}
		\resitem{Selected components (video DAC, ADC) and created schematics in OrCAD.}
		\resitem{Implemented image processing functions and data control blocks in VHDL using an Altera ACEK1K FPGA.  Learned VHDL and MAX+PlusII development tool on my own time.}
		%\resitem{Simulated the various blocks using MAX+PlusII simulation tool.}
	\end{itemize}

\item
	\ressubheading{Analog Circuit Design and MOSFET Device Design}{}{Semiconductor Devices Course, EECE 480}{Sep. 2001 - Apr. 2002}
	\begin{itemize}
		\resitem{Designed a high-frequency cascode amplifier, simulated it using HSPICE, and did layout using Cadence Virtuoso Layout software.  Manufactured on a Gennum GA911 chip.}
		\resitem{Designed and simulated a deep sub-micron (~70 nm channel) MOSFET using MEDICI.}
	\end{itemize}

\item
	\ressubheading{Low-cost Optoelectronic Localizer}{}{Engineering Physics Project Lab, APSC 459}{Sep. 2000 - Apr. 2001}
	\begin{itemize}
		\resitem{Worked on the LoCOL (Low-cost Optoelectronic Localizer) project in a team of three.}
		\resitem{Programmed a PIC microcontroller to control the timing of the three CCD cameras.}
		\resitem{Designed power supply and re-built electrical circuits for the CCD sensors, processors.}
		%\resitem{Helped build an enclosure for optics and electronics in the Student Machine Shop.}
	\end{itemize}

\item
	\ressubheading{Other Projects}{}{UBC and at home}{1999-2000}
	\begin{itemize}
		\resitem{Designed and debugged a digital voltmeter using a Motorola 68000 processor.}
		\resitem{Added features to the digital voltmeter including scrolling text, and a warning buzzer, which won 3rd place in the IEEE Voltmeter Competition.}
		\resitem{Constructed and debugged a digital clock on a PCB for PHYS 159.}
		\resitem{Built an AM short-wave radio at home, on a 2" $\times$ 2.5'' piece of breadboard.}
	\end{itemize}

\end{itemize}

\resheading{Awards}
	\begin{tabular*}{6.5in}{l@{\extracolsep{\fill}}r}
		Faculty of Engineering Scholarship (\$2,300) & 2002\\
		Ontario Graduate Scholarship (OGS) (\$15,000) & 2002-2003\\
		Industrial NSERC Undergraduate Research Award (\$4500) & 2002\\
		UBC OSI (Outstanding Student Initiative) Entrance Scholarship (\$10,000) & 1997-2002\\
		Engineering Physics 50th Anniversary Scholarship (\$600) & 2001\\
		Anne. M. Mack Scholarship (\$500) & 2001\\
		NSERC Undergraduate Student Research Award (\$4000) & 2000\\
		United Food and Commercial Workers Union, Local 1518 Scholarship (\$1000) & 1998\\
		Top Senior Math Student Award & 1997\\
		B.C. Provincial Exam Scholarship (\$1000) & 1997\\
		B.C. Government Passport to Education (\$800) & 1997\\
		James Whiteside Elementary Parent Advisory Committee Award (\$200) & 1997\\
\end{tabular*}

\resheading{Skills}

\begin{description}
\item[Languages:]
C/C++, \LaTeX, Java, SPICE, MEDICI (TCAD), VHDL/VHDL-AMS, 68000 and PIC Assembly
\item[Operating Systems:]
Linux (Debian), Solaris, UNIX, MacOS X, Windows 95/98/NT/2000/XP
\item[Applications:]
Mathematica, MatLab, GNU Octave, LabVIEW, Cadence, \LaTeX, OpenOffice, MS Office XP, OrCAD schematic capture \& PCB layout, Altera MAX+PlusII VHDL FPGA Design
\item[Lab Skils:]
Digital/Analog Scopes, Spectrum Analyzer, Function Generators
\item[Fab Skills:]
PECVD and sputtering deposition, UV lithography, wet etch, dry etch (RIE), mask aligner, step profiler, ellipsometry, infrared spectroscopy, x-ray diffraction
\item[Miscellaneous:]
software configuration management, strong verbal and written communication skills, excellent troubleshooting and debugging skills, exceptional problem solving skills, good teams skills
\end{description}

\resheading{Interests}

\begin{description}
\item[Academic:] Solid state devices,  nanotechnology, photonics, microcontrollers, RF/wireless
\item[Sports:] Playing hockey and swimming
\item[Computers:] Currently maintain two official Debian Linux packages, Mozilla beta tester, enjoy using and learning Linux systems, Building electronics projects at home, and writing JAVA software
\item[Musical:] Playing guitar and piano
\item[Membership:] Student member of IEEE since 1998, Materials Research Society member since 2002
\item[Other:] Reading novels
\end{description}

\end{document}
