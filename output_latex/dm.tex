\documentclass[a4paper,10pt,english]{article}
\usepackage[utf8ttf,utf8]{inputenc}
\usepackage{ttfucs}
\usepackage[T1]{fontenc}
\usepackage{babel}
\usepackage{times}
\usepackage[Bjarne]{fncychap}
\usepackage{sphinx}

\usepackage{hyperref}
\usepackage{fontspec}
\usepackage{xunicode}
\usepackage{xltxtra}
\XeTeXlinebreaklocale "zh"
\XeTeXlinebreakskip = 0pt plus 1pt

\setsansfont{WenQuanYi Zen Hei}
\setromanfont{WenQuanYi Zen Hei}

\title{数据挖掘概念与技术笔记整理}
\author{liz\footnote{Email: shengyan1985@gmail.com}}

\begin{document}
\maketitle
\tableofcontents

\section{基本概念整理}
\subsection{基本定义}
\begin{enumerate}
\item
\textbf{数据仓库 Data Warehouse}:
一种数据存储仓库的系统结构,一种多个异构数据源在单个站点以统一的模式组织的储存库,以支持管理决策。
\paragraph{}
这是一个比较抽象的概念,刚开始读n遍估计也不知道到底什么是数据仓库。
我的理解是,集成了多种数据源,可以是文本,结构化的数据库,任何东西,按照一定策略组织起来,其
中包含对这些数据的描述,包含时间信息等。之后,可以在这巨大的数据上进行挖掘。

\item
\textbf{数据挖掘 Data Mining}: 从大量数据中提取或“挖掘”知识。
\paragraph{}
另外一个说法就是,知识发现(Knowledge Discovery in
Database),可以看成是广义的数据挖掘,因为KDD还包括数据预处理,数据变换,评估,知识表示等,
数据挖掘只是其中的一个核心部分。而数据挖掘对应的数据源,不仅可以是精心组织的数据仓库中的数据,
也可以是动态的数据,如数据流,随时间改变而改变的数据等等,可以说是任何数据。

\item
\textbf{数据挖掘主要包括哪些功能}:
\begin{itemize}
\item
概念/类的描述:特征化和区分
\paragraph{}
这里主要是如何对数据描述以更好的表达数据包含的信息,如对一个文本,如何提取特征词,
如何组织其间结构以便既能精简文本原数据量又能准确的表达出文本内容。
\item
挖掘频繁模式,关联规则和相关规则。包括频繁项集,频繁序列模式,频繁结构模式。
\paragraph{}
像关联规则,现有许多网站都有这方面的相关应用,比较成熟。其中,有两个度量在关联规则中非常重要,
分别是:支持度和置信度。下述。
\item
分类与预测Classfication\&Predication
\paragraph{}
分类和预测都是一种有监督的学习,就是说利用已有的知识,如已经知道每个对象属于的类标号,进行学习得到一个分类模型,
之后再利用这个分类器对未知的对象进行分类,即给定这个未知对象的类标号。这里的类标号本质上也是一个属性,称之为决策属性,
而对应的,其他属性称为分类属性,也就是说针对已知对象(训练集)上,寻找多个分类属性和决策属性之间的关系。
\item
聚类Cluster
\paragraph{}
聚类是一种无监督的学习,属性不区分分类属性还是决策属性,因为事先我们不知道每个对象是属于哪个类,需要自动发掘他们之间的类别
,类别与类别之间的边界等。所谓物以类聚,人与群分,现实生活中,仔细想想,也无处不在聚类啦。。。
\item
还有其他的,如离群点分析Outlier Mining和演变分析Evolution Analysis。
\end{itemize}
\paragraph{}
可以说,数据挖掘涉及的范围是超级广的,层面也是非常多的。所以非常经常的看到好多大学实验室研究的内容都是属于数据挖掘这个大大范围的。

\item
\textbf{支持度Support}:满足规则$X \Rightarrow
Y$的事物在数据库中所占的百分比。主要针对事务数据库。 记:$Support(X
\Rightarrow Y)=P(X \bigcup Y)$
\item
\textbf{置信度Confidence}:评估发现的规则的确定性程度。记:$Support(X
\Rightarrow Y)=P(X|Y)$
\paragraph{}
规则的支持度和置信度是规则兴趣度的两种度量,分别反映所发现的规则的有用性和确定性。一般设置最小支持度阈值minsup,最小置信度阈值minconf。
同时满足minsup和minconf的规则成为强规则。

\item
\textbf{频繁模式Frequent
Pattern}:频繁的出现在数据集中的模式,包括项集,子序列,子结构。其中的频繁项集是指频繁的同时出现在交易数据集中的东西的集合。
频繁序列是频繁的同时出现在数据集中的序列模式。还有子结构,子图,子树,子格等等这些是结构模式。

\item
\textbf{项集}:项的集合,k项集表示有k个元素。\textbf{项集频率}:支持度计数,包括该项集的事务数。频繁k项集记为$L_{k}$。
$Confidence(A \Rightarrow B)=P(B|A)=\frac{Support(A \bigcup
B)}{Support(A)}$ 所以,挖掘关联规则的问题可归结为挖掘频繁项集。
一个项集是频繁的,则它的每个子集也是频繁的。

\item
\textbf{闭频繁项集}:不存在真超项集Y,使Y与X在S中有相同的支持度计数,则称X在S中是封闭的。
\textbf{极大频繁项集}:不存在超项集Y,使$$X \subseteqq Y $$
并且Y在S中是频繁的。
\end{enumerate}

\subsection{基本定义}
通天塔

\end{document}
